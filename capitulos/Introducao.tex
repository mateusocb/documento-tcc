% Introdução
\chapter{Introdução}\label{cap:introducao}

A introdução é a parte inicial do texto e que possibilita uma visão geral de
todo o trabalho, devendo constar a delimitação do assunto tratado, objetivos da
pesquisa, motivação para o desenvolvimento da mesma e outros elementos
necessários para situar o tema do trabalho.

Atualmente o ensino de programação e robótica voltado aos jovens e crianças vêm
ganhando espaço em praticamente todo o mundo, por outro lado existe a fama da
alta complexidade envolvida com esses dois assuntos. Por causa dessa alta
complexidade nasceu uma grande necessidade pela criação de metodologias e
estratégias mais eficientes para um ensino de programação e também de robótica.
Nesse meio de busca de novas estratégias de ensino de programação, nasceu a
programação tangível que busca contornar os problemas da alta complexidade
ligada à informática, com o príncio de modificar a forma de programar que
normalmente é usando recursos virutais e intocáveis, por objetos que podem ser
tocados e vistos no mundo real.

Seguindo o princípio da programação tangível, a Google criou uma platarforma
que inclui um projeto que possibilita a criação de programas através da
combinação de pequenos módulos no formato de blocos que se encaixam, onde cada
bloco tem um comportamento especifico e podem ser combinados de diversas formas
produzindo novos comportamentos mais complexos, esse projeto foi nomeado de
Project Bloks. Dentro dessa mesma plataforma existe também um pequeno robô que
consegue produzir desenhos usando uma caneta, esse robô é chamado de mirobot, a
programção das ações do mirobot pode ser feita de diversas formas, que vai
desde a combinação dos módulos do já citado Project Bloks, até a construção
desses comportamentos por meio de linguagens de programação como: javacript,
python e scratch.

Outra dificuldade no aprendizado de programação, além do alto nível de
abstração ligado a computação, são também as sintaxes complexas nas linguagens
de programação. E isso se agrava mais ainda quando a linguagem é baseada em um
idioma no qual não é do seu conhecimento. Pensando nisso, foi que no ano de
2011 iniciou-se no Instituto Federal de Educação, Ciências e Tecnologia do Rio
grande do Norte (IFRN), por meio do professor Leonardo Reis Lucena e alunos
bolsitas, a projeto da linguagem potigol. A proposta do potigol é ser uma
linguagem de programação que tem sua sintaxe baseada no idioma português, e que
com isso auxilie o ensino de programação para os falantes desse mesmo idioma.

Esse trabalho busca juntar essas duas experiências ja citadas, de forma que
elas possam entrar em sintonia e trazer um resultado ainda maior, levando em
conta que elas têm o mesmo objetivo final. Dessa forma o objetivo principal do
trabalho é permitir que exista uma alternativa além das linguagens já
suportadas para a programação das ações do mirobot, a possibilidadede usar a
linguagem potigol.

\section{Objetivos}

Nesta seção são definidos os objetivos gerais e específicos do trabalho.

\subsection{Objetivos Gerais}

Este trabalho tem por objetivo geral implementar uma biblioteca que permita
programar na linguagem Potigol comportamentos de um robô Mirobot.

\subsection{Objetivos Específicos}

Fazem parte dos objetivos específicos deste trabalho:

\begin{itemize} \item Descrever o protocolo de troca de mensagens do robô
    Mirobot; \item Desenvolver a biblioteca mirobot-poti na linguagem de
    progração Java; \item Implementar exemplo de utilização da biblioteca
      Mirobot-Poti em Potigol utilizando um simulador para o robô;
\end{itemize}


\section{Metodologia}

Esse trabalho seguiu as seguintes etapas, primeiramente foram pesquisadas e
entendidas as ferramentas já existentes desenvolvidas pela própria Google, que
é a mesma entidade que desenvolveu o robô mirobot, entre elas estão:
mirobot-sim, um simulador do robô mirobot escrito em javascript; mirobot-py e
mirobot-js, que são implementações das bibliotecas para comunicação com o robô
nas respectivas linguagens: python e javascript. Após essa etapa foram
definidos os nomes dos métodos da biblioteca mirobot-poti que seriam
posteriormente implementados, esses métodos são responsáveis por comandar os
comportamentos do mirobot, essa nomeação foi feita com base nas bibliotecas já
citadas: mirobot-py e mirobot-js. A próxima etapa foi entender como fazer a
importação de uma biblioteca externa na linguagem potigol, a qual foi
compreendida através de simples testes compilando códigos da linguaguem Java e
importando o resultado em códigos no potigol. Depois do sucesso da validação
referente a importação de uma biblioteca externa no potigol, foi feita uma
segunda validação com o mesmo propósito, sendo que agora a tentativa de
importação foi feita com uma biblioteca protótipo que faz uma simples
comunicação com o simulador do mirobot. E por fim, levando como base o sucesso
das validações já citadas, foram implementados todos os métodos anteriormente
nomeados e criado um exemplo de uso da utilização da biblioteca mirobot-poti.

\section{Delimitação do trabalho}

Devido a dificuldades na importação do robô, foi utilizado um simulador ao
invés do próprio robô. Por outro lado, a ausência do robô não teve grande
impacto no desenvolvimento da biblioteca já que existia todo um ambiente pronto
pra suportar os testes necessários.


\section{Organização do trabalho}

O trabalho esta organizado na seguinte estrutura: o
\hyperref[cap:introducao]{primeiro capítulo}, com uma introdução sobre o
trabalho; o \hyperref[cap:referencial]{segundo capítulo} com referencial
teórico utilizado relativo ao mirobot e a liguagem de programação potigol, o
\hyperref[cap:descricao]{teceiro capítulo} que contém a descrição da
implementação e os testes, e por fim o \hyperref[cap:consideracoes]{quarto
capítulo} com as considerações finais do trabalho.
