% Introdução
\chapter{Introdução}
\label{cap:introducao}

Atualmente o ensino de programação e robótica voltado aos jovens e crianças vêm
ganhando espaço em praticamente todo o mundo, no entanto existe a fama da
alta complexidade envolvida com esses dois assuntos. Por causa dessa alta
complexidade, pesquisadores e professores procuraram metodologias e
estratégias diferentes para um ensino de programação bem como de robótica.
Uma das novas estratégias de ensino programação é baseada na
programação tangível~\cite{Horn2007} que busca contornar os problemas da alta complexidade ligada à informática, com o princípio de modificar a forma de programar que
normalmente é usando recursos virtuais e intocáveis, por objetos que podem ser
tocados e vistos no mundo real.


O Google mantém uma plataforma baseada na programação tangível chamada de \textit{Project Bloks} (\url{https://projectbloks.withgoogle.com}) 
que possibilita a criação de programas através da
combinação de pequenos módulos no formato de blocos que se encaixam. 
Cada bloco tem um comportamento específico e pode ser combinado com outros blocos de diversas formas
produzindo novos comportamentos mais complexos.
Dentro dessa mesma plataforma também existe um pequeno robô que consegue produzir desenhos usando uma caneta, chamado de mirobot. A programação das ações do mirobot pode ser feita de diversas formas, que vai desde a combinação dos módulos do Project Bloks, até a construção
desses comportamentos por meio de linguagens de programação como: JavaScript,
Python e Scratch.

Outra dificuldade no aprendizado de programação, além do alto nível de
abstração ligado à computação, são as sintaxes complexas nas linguagens
de programação. Principalmente quando a linguagem de programação é baseada em um
idioma diferente da língua materna dos alunos. Pensando nisso, foi que no ano de
2011 iniciou-se no Instituto Federal de Educação, Ciências e Tecnologia do Rio
grande do Norte (IFRN), por meio do professor Leonardo Reis Lucena e alunos
bolsistas, a projeto da linguagem Potigol~\cite{potigol}. A proposta do Potigol é ser uma
linguagem de programação que tem sua sintaxe baseada no idioma português, multiparadigma, com suporte de softwares para o ensino de programação dos falantes desse mesmo idioma.

Este trabalho busca juntar essas duas experiências já citadas, de forma que
elas possam entrar em sintonia, levando em
conta que elas têm o mesmo objetivo final, auxilar no ensino de programação. 
Dessa forma o objetivo principal do
trabalho é permitir que exista uma alternativa além das linguagens já
suportadas para a programação das ações do mirobot.

\section{Objetivos}

\subsection{Objetivo Geral}

Este trabalho tem por objetivo geral implementar uma biblioteca que permita
programar na linguagem Potigol comportamentos de um robô Mirobot.

\subsection{Objetivos Específicos}

Fazem parte dos objetivos específicos deste trabalho:

\begin{itemize}
  \item Descrever o protocolo de troca de mensagens do robô Mirobot;
  \item Desenvolver a biblioteca mirobot-poti na linguagem de
    programação Java;
  \item Implementar exemplo de utilização da biblioteca
      Mirobot-Poti em Potigol utilizando um simulador para o robô;
\end{itemize}


\section{Metodologia}

Esse trabalho seguiu as seguintes etapas. Primeiramente foram pesquisadas e
entendidas as ferramentas já existentes desenvolvidos pelo Google, entre elas estão:
mirobot-sim, um simulador do robô mirobot escrito em Javascript; mirobot-py e
mirobot-js, que são implementações das bibliotecas para comunicação com o robô
nas respectivas linguagens: Python e Javascript. 
Após essa etapa foi
definida a interface(assinatura dos métodos) da biblioteca mirobot-poti que seria
posteriormente implementada pora comandar os
comportamentos do mirobot, a qual a nomeação foi feita com base no protocolo do mirobot diponibilizado em: \url{http://learn.mirobot.io/docs/understanding-the-mirobot-protocol/}.
A próxima etapa foi entender como fazer a
importação de uma biblioteca externa na linguagem Potigol, a qual foi
compreendida através de simples testes compilando códigos da linguagem Java e
importando o resultado em códigos no Potigol. 
Depois foi feita uma
segunda validação com a biblioteca se comunicando com o simulador do mirobot. 
E por fim, foram implementados todos os métodos anteriormente
nomeados e criado um exemplo de uso da utilização da biblioteca mirobot-poti.

\section{Delimitação do trabalho}

Devido a dificuldades na importação do robô, foi utilizado um simulador ao
invés do próprio robô. Por outro lado, a ausência do robô não teve 
impacto no desenvolvimento da biblioteca já que existia todo um ambiente pronto
para suportar os testes necessários.


\section{Organização do trabalho}

O trabalho está organizado na seguinte estrutura: este primeiro capítulo, com uma introdução sobre o trabalho; o segundo capítulo com o referencial teórico utilizado relativo ao mirobot e a linguagem de programação Potigol, o terceiro capítulo que contém a descrição da implementação e os testes, e por fim o quarto capítulo com as considerações finais do trabalho.
