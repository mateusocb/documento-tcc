% Consideracoes finais
\chapter{Considerações finais}
\label{cap:consideracoes}


% As considerações finais formam a parte final (fechamento) do texto, sendo dito
% de forma resumida (1) o que foi desenvolvido no presente trabalho e quais os
% resultados do mesmo, (2) o que se pôde concluir após o desenvolvimento bem
% como as principais contribuições do trabalho, e (3) perspectivas para o
% desenvolvimento de trabalhos futuros, como listado nos exemplos de seção
% abaixo.  O texto referente às considerações finais do autor deve salientar a
% extensão e os resultados da contribuição do trabalho e os argumentos
% utilizados estar baseados em dados comprovados e fundamentados nos resultados
% e na discussão do texto, contendo deduções lógicas correspondentes aos
% objetivos do trabalho, propostos inicialmente.

Com a término da fase de desenvolvimento desse trabalho foi possível alcançar o
objetivo geral de desenvolver uma biblioteca para programar comportamentos de
um robô mirobot em potigol. Em relação aos objetivos específicos, todos foram
alcançados como no planjeado: foi descrito o protocolo usado no mirobot,
desenvolvido a biblioteca mirobot-poti e implementado um exemplo em potigol do
uso da biblioteca. O produto desse trabalho foi a biblioteca que aumenta as
possíbilidades de linguagens que podem ser utilizadas para programar
comportamentos do mirobot, só que com o detalhe que o potigol oferece um
recurso diferencial, que é a vantagem de uso do potigol por parte de pessoas
que falam português mas não entendem o inglês.


\section{Principais contribuições}

\section{Limitações}

A principal limitação que foi enfrentada, além da ausência do robô físico para
os testes, foi que a biblioteca mirobot-poti não foi implementada com noção de
threads diferente da mirobot-py. Por causa dessa limitação não foi possível
implementar a noção da fila de comandos que deveria ser controlada pela
pela próprio biblioteca mirobot-poti.


\section{Trabalhos futuros}

Ainda há uma grande necessidade da adoção novas estratégias e metodologias no
ensino de assuntos que são levados como difíceis e complexos, como é o exemplo
da programação de computadores. O mirobot-poti é apenas um passo para que se
alcance a possibilidade de existir a programação tangível como forma
alternativa de ensino no brasil. Algumas possibilidades de continuar a
desenvolver nessa área são: o desenvolvimento de um rôbo similar ao mirobot de
baixo custo e complexidade de forma que seja acessível a qualquer um
construí-lo, e desenvolver uma alternativa acessível para programação tangível
com blocos. Como forma de continuídade do trabalho que foi desenvolvido, há a
possibilidade de desenvolver um terminal que consiga enviar os comandos ao
mirobot interativamente. 
