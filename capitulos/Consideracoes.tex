% Consideracoes finais
\chapter{Considerações finais}
\label{cap:consideracoes}


% As considerações finais formam a parte final (fechamento) do texto, sendo dito
% de forma resumida (1) o que foi desenvolvido no presente trabalho e quais os
% resultados do mesmo, (2) o que se pôde concluir após o desenvolvimento bem
% como as principais contribuições do trabalho, e (3) perspectivas para o
% desenvolvimento de trabalhos futuros, como listado nos exemplos de seção
% abaixo.  O texto referente às considerações finais do autor deve salientar a
% extensão e os resultados da contribuição do trabalho e os argumentos
% utilizados estar baseados em dados comprovados e fundamentados nos resultados
% e na discussão do texto, contendo deduções lógicas correspondentes aos
% objetivos do trabalho, propostos inicialmente.

Com a término da fase de desenvolvimento desse trabalho foi possível alcançar o
objetivo geral, permitir programar comportamentos de
um robô mirobot em linguagem potigol. 
Em relação aos objetivos específicos, todos foram
alcançados como o planjeado: foi descrito o protocolo usado no mirobot,
desenvolvido a biblioteca mirobot-poti e implementado um exemplo em potigol do
uso da biblioteca.


\section{Principais contribuições}

A contribuição pessoal foi no aprendizado de novas tecnologias, tais como: o websocket, desenvolvimento de uma biblioteca na liguagem java e o contato com a linguagem potigol. Além da oportunidade de interagir com novas áreas como foi o caso da programação tangível. 

A principal contribuição para o projeto Potigol é permitir criar uma "sala de aula" mais interativa.
Os alunos irão programar em uma linguagem em português comportamentos de um robô, tal como desenvolver a lógica para desenho de um quadrado e de outros objetos geométricos mais complexos.

\section{Limitações e dificuldades}

A limitação principal do projeto foi a ausência de robô para teste real devido a problema na importação do equipamento.
Além da ausência do robô físico para
os testes, uma outra limitação foi que a biblioteca mirobot-poti não foi implementada com uso de
threads. Uma biblioteca que usou threads foi a mirobot-py.
Por causa dessa limitação, todos os comportamentos são síncronos, não sendo possível o enfileiramento de comandos controlados pela própria biblioteca mirobot-poti.

Um obstáculo no projeto foi entender como o Potigol usa bibliotecas externas e assim poder fazer o mirobot-poti uma alternativa no ensino, Ou seja, integrar na programação a biblioteca apenas quando necessário.
Foram realizadas algumas interações com a equipe do projeto Potigol para que a biblioteca fosse usada apenas especificando como biblioteca externa (uso de \textit{classpath}). 

Ficou o \textit{bug} na saída dos comandos com o uso específico do comando java -cp potigol.jar:mirobot.jar:. br.edu.ifrn.potigol.Principal -w UsandoMirobot.poti, que resultou na mensagem: \textit{aguarde...}, após um tempo a mensagem era apagada e não tinha nenhum resultado de sucesso nem de erro.

%% não lembro o comando... tem de ver no trello


\section{Trabalhos futuros}

A partir deste trabalho, existem algumas possibilidades de continuação no desenvolvimento de software, desenvolvimento de rôbos bem como na avaliação do ensino de programação.
Quanto ao desenvolvimento de software, é possível destacar:

\begin{itemize}
    \item Desenvolvimento de um \textit{shell} permita enviar comandos diretamente ao um mirobot, mirobot-poti-shell;
    \item Extensão do mirobot-poti-shell para permitir comandos potigol, inclusive a carga de bibliotecas em tempo de execução;
    \item Desenvolver outras bibliotecas que permitam manipulação de outros rôbos, como por exemplo o rôbo músico do Google ou carros rôbos usados nas olimpíadas de robótica nacionais.
\end{itemize}


No processo de ensino de programação, é importante fazer testes controlados em laboratório para avaliar a efetividade do uso de potigol para manipular o rôbo assim como para o desenvolvimento de lógica de programação.
