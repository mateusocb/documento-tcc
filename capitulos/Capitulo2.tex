% Capítulo 2
\chapter{Mirobot e Potigol: tecnologias alternativas para o ensino de programação}
\label{cap:referencial}

\section{Mirobot: uma proposta de uso de robôs para ensino de lógica de
programação}
\label{sec:mirobot}

Um dos principais objetivos da programação tangível é tornar o que é complexo e
virutal em algo simples que pode ser visto e manuseado facilmente no mundo
real~\cite{Horn2008, Horn2007, McNerney2000}. Esse mesmo pensamento pode ser estendido da seguinte forma: durante o aprendizado de programação é comum que os alunos apliquem seus
conhecimentos para manipular recursos virtuais como: números, cadeias de
caracteres (strings) e valores booleanos. 
Como forma de alternativa ao uso dos
objetos virtuais, existe a possibilidade de se usar robôs e programar as suas
ações. 
Tenta-se com isso tornar mais intuitivo para visualizar e aprender a
lógica de programação, principalmente para crianças, pois os resultados dos
seus programas serão mostrados em um mundo real e não em uma tela de computador.

Devido a esse cenário, a empresa Google desenvolveu o Mirobot. O Mirobot é em um robô
cujas ações podem ser programadas tanto através das
linguagens de programação: Scratch, Javascript e Python; como também através um módulo que
também implementa a noção de programação tangível chamado \textit{Project Bloks}. 
A proposta do mirobot é ser um robô que recebe comandos de uma determinada fonte
via Wi-Fi e ao realizar essas ações ele vai marcando o seu trajeto com uma
caneta, resultando em um desenho.
Entre as principais ações que ele realiza
estão: se movimentar para frente ou para atrás, girar para esquerda ou para
direita, descer ou levantar a caneta (possibilitando com que ele desenhe ou não
seu trajeto quando desejado).


\subsection{Resumo da arquitetura do Mirobot}

% hardware microcontrolador, motor parágrafo para explicar a parte física do
% mirobot.

O Mirobot é composto basicamente por um chassi de madeira que é formado por:
duas rodas responsáveis pela sua movimentação, uma terceira roda no formato
esférico (que ajuda na sustentação e movimentação do robô) e um orifício onde é
possível fazer o encaixe de uma caneta.
Já na parte dentro do seu corpo,
existem dois motores de passo, cada um responsável por uma das rodas, os quais
são encarregados pela sua movimentação. 
Além desses dois motores, ainda há um
servomoto que fica responsável pelo acionamento da caneta. 
Na parte de controle
ele contém um arduíno nano responsável pelo acionamento dos motores, e a
comunicação via Wi-Fi e o tratamento dos dados fica por conta de um controlador
principal, o qual é desenvolvido pela mesma empresa do mirobot.

% parágrafo para explicar a parte comunicação do mirobot.
% software websocket
A comunicação com o Mirobot é feita via Wi-Fi e ele trabalha com a tecnologia
chamada websocket ~\cite{websocket2011}. O websocket consiste em uma tecnologia que oferece um meio
de comunicação bidirecional através de um único soquete, essa tecnologia
normalmente é usada em browsers web e servidores web. 
As mensagens que são
enviadas e recebidas pelo Mirobot estão no formado JSON (Javascript Object
Notation).

Todo esse projeto é mantido em um repositório no portal Github nos endereços: chassi em \url{https://github.com/mirobot/mirobot-chassis}; firmware arduino em \url{https://github.com/mirobot/mirobot-arduino} e o firmware wifi em \url{https://github.com/mirobot/mirobot-wifi}.

\subsection{Códigos e projetos do Mirobot}
\label{subsec:mirobotcodigos}

% Projetos que estão no Github

O Mirobot é rodeado por um grande ecossistema de projetos relacionados a ele no
repositório Github (\url{https://github.com/mirobot}), entre eles estão:

\begin{itemize}
    \item \textit{mirobot-sim}, que é um simulador
que tem a função de receber as mensagens em JSON via websocket e simular a
parte de comunicação do robô;
    \item \textit{mirobot-py}, \texit{mirobot-js} e \texit{mirobot-scratch} que são
bibliotecas para o controle do mirobot para as respectivas linguagens Python,
Javascript e Scratch;
\end{itemize}

Além dos projetos no Github ainda existe uma gama de aplicativos web com
plataformas de interação com o mirobot, podem ser encontradas em:
\url{http://apps.mirobot.io/}. Para cada uma das linguagens de programação há uma aplicação web que possibilita comunicação com o robô, em todas
essas aplicações, além da possibilidade de se conectar com o robô físico e
enviar os comandos, é possível acompanhar o seu trajeto por meio de um
simulador.  Ainda existem outros aplicativos relacionados a comunicação com o
mirobot, entre eles estão: point \& click, que é uma interface para definir o
trajeto do mirobot apenas usando clicks na tela de um simulador; e remote
control, que atua como um remoto possibilitando o envio das ações para o
mirobot por meio de botões na interface web.

\section{Linguagem Potigol uma linguagem em português para o ensino de lógica
de programação de computadores}
\label{sec:potigol}

No ano de 2011 no Instituto Federal de Educação, Ciências e Tecnologia (IFRN),
Campi Natal/Central (CNAT), foi iniciado o projeto da linguagem de programação
Potigol pelo professor Leonardo Reis Lucena e alunos bolsistas. A linguagem
nasceu com a proposta principal de auxiliar o ensino das disciplinas de
programação de computadores, e leva como principal característica a sua sintaxe
em português. Entre algumas outras características da linguagem estão: multiparadigma, tipagem estática com inferência de tipos, projetada para ser
usada por alunos iniciantes.

Atualmente a linguagem Potigol com tutorial e seus softwares utilitários estão disponíveis em \url{http://potigol.github.io} e seus projetos com seus códigos-fontes em \url{https://github.com/potigol/}.

%% Adicionar exemplo do potigol.
%% falta as referências sobre o potigol
%% 
