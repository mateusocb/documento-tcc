% Resumo
\begin{center}
	{\Large{\textbf{\ABNTtitulodata}}}
\end{center}

\vspace{1cm}

\begin{flushright}
  Autor: \ABNTautordata\\
  Orientador(a): Leonardo Ataide Minora, Mestre
\end{flushright}

\vspace{1cm}

\begin{center}
	\Large{\textsc{\textbf{Resumo}}}
\end{center}

\noindent O aumento de interesse pela área da programação voltada para jovens e
crianças vem com o aumento da necessidade de novas estratégias e metodologias
de ensino que se adequem a esse público. Além dessa necessidade, existe a
necessidade de meios alternativos para o ensino de programação nos cursos
superiores e técnicos, principalmente no contexto brasileiro já que a área de
informática esta diretamente ligado à língua inglesa, na qual nem todos os
alunos falantes do português tem domínio. Seguindo esses dois contextos nasce a
proposta desse trabalho, que é juntar uma forma alternativa de ensino de
programação, chamada de programação tangível, a qual nasceu para ser voltada ao
ensino de programação para prianças; com uma linguagem de programação em
português desenvolvida para o ensino de lógica de programação, o potigol. Essa
junção vai ser feita através da construção de uma biblioteca para a linguagem
potigol, que possibilite programar as ações de um robô chamado de mirobot.

% O resumo deve apresentar de forma concisa os pontos relevantes de um
% texto, fornecendo uma visão rápida e clara do conteúdo e das conclusões do
% trabalho. O texto, redigido na forma impessoal do verbo, é constituído de uma
% sequência de frases concisas e objetivas e não de uma simples enumeração de
% tópicos, não ultrapassando 500 palavras, seguido, logo abaixo, das palavras
% representativas do conteúdo do trabalho, isto é, palavras-chave e/ou
% descritores. Por fim, deve-se evitar, na redação do resumo, o uso de
% parágrafos
% (em geral resumos são escritos em parágrafo único), bem como de fórmulas,
% diagramas e símbolos, optando-se, quando necessário, pela transcrição na forma
% extensa, além de não incluir citações bibliográficas.

\noindent\textit{Palavras-chave}: programação tangível, robô mirobot, potigol.
